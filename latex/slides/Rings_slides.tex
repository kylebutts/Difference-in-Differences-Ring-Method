\documentclass[aspectratio=43]{beamer}
% \documentclass[aspectratio=169]{beamer}

% Title --------------------------------------------
\title{Treatment Effect Estimation with Geocoded Microdata}
\subtitle{}
\date{\today}
\author{Kyle Butts}

\input{preamble.tex}

\newcommand{\dist}{\text{Dist}}

% Set-up Bibliography ------------------------------
\addbibresource{references.bib}

\begin{document}

% ------------------------------------------------------------------------------
\maketitle
% ------------------------------------------------------------------------------

\begin{frame}{Motivation}

    Treatment often isn't assigned to groups of individuals (e.g. counties), but rather to specific locations

    \begin{itemize} 
        \item Environmental: Local pollutants \citepcolor{Marcus_2021,Currie_Davis_Greenstone_Walker_2015}, shale gas discovery \citepcolor{Muehlenbachs_Spiller_Timmins_2015}
    
        \item Urban: foreclosures \citepcolor{Gerardi_Rosenblatt_Willen_Yao_2015,Campbell_Giglio_Pathak_2011}, abandon lot cleanups \citepcolor{South_Hohl_Kondo_2018}, apartment construction \citepcolor{Asquith_Mast_Reed_2021} 
    \end{itemize}
\end{frame}

\setbeamercovered{transparent}

\begin{frame}{Identification of Treatment Effects}
    Difference-in-differences is often used to estimate treatment effects. How do you choose a control group?
    \vspace{2.5mm}
    
    \begin{enumerate}
        \item<1> Find alternative treatment locations to serve as a control group. For example propensity score match census blocks and use observations in blocks likely to receive treatment but did not.
            
            \begin{itemize}
                \item Problem if areas actually treated are targeted due to contemporaneous shocks to trends.
            \end{itemize}
        
        \item<2> Compare observations very near to treatment, ``the treated'', to observations just slightly further away, ``the control''.
            
            \begin{itemize}
                \item Identification allows treatment to be targeted due to `neighborhood` trends but not targeted for differential trends `within` the neighborhood.
            \end{itemize} 
    \end{enumerate}
\end{frame}


% Example Quotes
\begin{frame}{Examples of Identification Discussion}
    \citetcolor{Linden_Rockoff_2008} look at the arrival of sexual offenders on neighborhood prices.They compare home sales \textbf{within 0.1 mile (about 2 city blocks) to home sales between 0.1 - 0.3 miles}. 

    \vspace{2.5mm}
    \begin{quote}
        ``This framework would be compromised only if sex offenders
        consistently moved into properties near which a localized disamenity was likely to emerge.''
    \end{quote}
\end{frame}


\begin{frame}{Examples of Identification Discussion}
    \citetcolor{Marcus_2021} considers leaking underground petroleum storage tanks that affect hyper-local drinking water. Look at long-run outcomes of children exposed to petroleum pollution

    \vspace{2.5mm}
    \begin{quote}
        ``I find that low-SES mothers are more likely to live near pollution sites. ... The remaining threat to identification is time-varying unobservable characteristics (e.g. local economic conditions) that vary systematically with the observed leak timing. To address
        this concern, I compare mothers \textbf{within two small radii of the leaking site, 300 and 600 meters.}''
    \end{quote}
\end{frame}


\begin{frame}{Problem} 
    The central problem is how do you chose the radii of the two rings?

    \begin{itemize}
        \item Does treatment effects stop at 0.1 miles? 0.2 miles? 0.3 miles?
        \item How far constitutes the "neighborhood" that control units can be in?
    \end{itemize}
\end{frame}

\begin{frame}{Contribution}
    \begin{enumerate}
        \item I formalize the identification strategy in an econometric framework.
        
        \item Define what assumptions are needed to identify estimands: `parallel trends' within control ring and correct treatment effect cutoff distance.
        
        \item I propose a new estimator that relaxes the correct treatment effect cutoff assumption using new work on non-parametric series estimators \citetcolor{Cattaneo_Crump_Farrell_Feng_2019,Cattaneo_Farrell_Feng_2019}.
    \end{enumerate}
\end{frame}


% ----------------------------------------------------------------------------
\section{Example of Problems}
% ----------------------------------------------------------------------------

\imageframe{../../figures/slides-example_dgp.pdf}
\imageframe{../../figures/slides-example_a.pdf}
\imageframe{../../figures/slides-example_b.pdf}
\imageframe{../../figures/slides-example_c.pdf}
\imageframe{../../figures/slides-example_d.pdf}

% ----------------------------------------------------------------------------
\section{Theory}
% ----------------------------------------------------------------------------

\begin{frame}{Model for Outcomes}{Setup}
    Units $i$ have locations $(x_i, y_i)$. Treatment turns on between the two periods $t = 0, 1$ at point $(\bar{x}, \bar{y})$. Therefore units vary in $\dist_i$, their distance to treatment. We have an iid panel sample. 


    \vspace{2.5mm}
    Outcomes are given by:

    $$
    Y_{it} = \underbrace{\color{coral} \tau(\dist_i) \one_{t = 1}}_{\text{Treatment Effect Curve}} + \mu_i + \underbrace{\color{kelly}\lambda(\dist_i) \one_{t=1}}_{\text{Counterfactual Trend}} + \varepsilon_{it},
    $$

    \vspace{2.5mm}
    where we assume $\varepsilon_{it} \perp \dist_i$.
\end{frame}

\begin{frame}{Model}{Estimand}
    Estimand of interest: 
    $$
    \bar{\tau} = \condexpec{ \tau(\dist_i) }{ \tau(\dist_i) > 0 }
    $$

    This is just the average treatment effect on those affected by treatment.
\end{frame}

\begin{frame}{Model}{First-Differences}
    Taking first differences, we have:

    $$
    \Delta Y_{it} = { \color{coral} \tau(\dist_i)} + {\color{kelly} \lambda(\dist_i)} + \Delta \varepsilon_{it}
    $$

    With no additional assumptions, the {\color{coral} treatment effect curve} and the {\color{kelly} counterfactual trend} are not seperately identifiable.
\end{frame}

\begin{frame}{Model}{Assumptions}
    \textbf{\color{alice} Assumption:} {\color{asher} Local Parallel Trends}

    For a distance $d_c$, local parallel trends hold if ${\color{kelly} \lambda}$ is constant for $0 < d < d_c$.

    \vspace{5mm}
    \textbf{\color{alice} Assumption:} {\color{asher} Average Parallel Trends}

    For a distance $d_c$ and $d_t$, average parallel trends hold if

    $$\condexpec{{\color{kelly} \lambda_d}}{0 \leq d \leq d_t} = \condexpec{{\color{kelly} \lambda_d}}{d_t < d \leq d_c}$$

    \pause
    \begin{itemize}
        \item {\color{asher} Average Parallel Trends} is a milder assumption, but usually researchers justify the method by {\color{asher} Local Parallel Trends}.
    \end{itemize}
\end{frame}

\begin{frame}{Model}{Assumptions}
    \textbf{\color{alice} Assumption:} {\color{asher} Correct Treatment Effect Cutoff}

    A distance $d_t$ satisfies this assumption if for all $d \leq d_t$, ${\color{coral} \tau(d)} > 0$ and for all $d > d_t$, ${\color{coral} \tau(d)} = 0$.

    \pause
    \begin{itemize}
        \item This is difficult to know!!!
        
        \item Later on, I show how to relax this assumption under {\color{asher} Local Parallel Trends}
    \end{itemize}
\end{frame}


\begin{frame}{Estimator}
    For a given $d_t$ and $d_c$, we define the treated and control groups as:
    
    $$\mathcal{D}_t = \{ i \ : \ \dist_i \leq d_t \}; \quad \mathcal{D}_c = \{ i \ : \ d_t < \dist_i \leq d_c \}$$
    
    
    On the sample $\mathcal{D}_t \cup \mathcal{D}_c$, the following regression is run:
    
    $$
        \Delta Y_{it} = \beta_0 + \beta_1 \one_{i \in D_t} + u_{it}
    $$

    $\hat{\beta}_1$ is the difference-in-differences estimate
    
\end{frame}

\begin{frame}{Identification}{Decomposition of Ring Method}

    \begin{enumerate}
        \only<1>{
        \item[(i)]The estimate of $\beta_1$ has the following expectation:
        {\footnotesize
            \[ 
                \expec{\hat{\beta}_1} = 
                \underbrace{\condexpec{\tau(\dist)}{\mathcal{D}_t} - \condexpec{\tau(\dist)}{\mathcal{D}_c} }_{\text{Difference in Treatment Effects}} 
                + \underbrace{\condexpec{\lambda(\dist)}{\mathcal{D}_t} - \condexpec{\lambda(\dist)}{\mathcal{D}_c} }_{\text{Difference in Trends}}.
            \]
        }
        }
        
        \only<2>{
        \item[(ii)] More, if $d_c$ satisfies {\color{asher} Local Parallel Trends} (or {\color{asher} Average Parallel Trends}), then
        {\footnotesize
            \[ 
                \expec{\hat{\beta}_1} = 
                \underbrace{\condexpec{\tau(\dist)}{\mathcal{D}_t} - \condexpec{\tau(\dist)}{\mathcal{D}_c} }_{\text{Difference in Treatment Effects}}.
            \] 
        }
        }
        
        \only<3>{
        \item[(iii)]<3> If $d_c$ satisfies {\color{asher} Local Parallel Trends}(or {\color{asher} Average Parallel Trends}) and $d_t$ is the correct distance cutoff, then
        {\footnotesize
            \[ 
                \expec{\hat{\beta}_1} = \bar{\tau}.
            \]
        }
        }
    \end{enumerate}
    
\end{frame}


% ----------------------------------------------------------------------------
\section{Improved Estimator}
% ----------------------------------------------------------------------------

\begin{frame}{Improved Estimator}{Difficulties with Assumptions}

In most cases, it is hard for a researcher to know ad-hoc what the correct $d_t$ is to satisfy the correct cutoff assumption.

\vspace{5mm}
There is a \emph{data-driven} way to estimate treatment effect curve, $\tau(\dist)$ without specifying $d_t$.
\begin{itemize}
    \item The method uses a non-parametric partition-based estimator proposed by \citetcolor{Cattaneo_Farrell_Feng_2019}.
\end{itemize} 
    
\end{frame}


\imageframe{../../figures/slides-example_did.pdf}
\imageframe{../../figures/slides-example_partition.pdf}

\begin{frame}{Improved Estimator}{Advantages}

    \begin{enumerate}
        \item Estimates a treatment effect curve rather than an average effect
        
        \begin{itemize}
            \item E.g. bus stop is built. Negative effects very close, positive effects slightly further away. Average effect $\approx 0$.
        \end{itemize}

        \pause
        \item Gives an informal visual test of {\color{asher} Local Parallel Trends}
  
        \begin{itemize}
            \item Curve should level off around zero
        \end{itemize}
    \end{enumerate}
\end{frame}

% ----------------------------------------------------------------------------
\section{Application}
% ----------------------------------------------------------------------------

\begin{frame}{Application}{\citet{Linden_Rockoff_2008}}
    \citetcolor{Linden_Rockoff_2008} look at the local effects on home prices of a registered sex offender moving into the neighborhood. 
    
    \begin{itemize}
        \item They compare homes within 0.1 miles of the offender's home to control units between 0.1 mile and 0.3 miles. 
        
        \emph{Is 0.1 mile the correct cutoff?}
        
        \item They assume that treatment effect is constant for being on the same block and being a few blocks away. 
        
        \emph{Is this a assumption correct?}
    \end{itemize}
\end{frame}

\begin{frame}
    \begin{figure}
        \caption{Effects of Offender Arrival on Home Prices \citep{Linden_Rockoff_2008}}
        \centering
        \includegraphics[width=\textwidth]{../../figures/linden_rockoff_did.pdf}
    \end{figure}
    \vspace{-5mm}
    \begin{figure}
        \vspace{-5mm}
        \centering
        \includegraphics[width=\textwidth]{../../figures/linden_rockoff.pdf}
    \end{figure}
\end{frame}



\begin{frame}{Conclusion}

The standard ``indicator'' version of the rings method requires knowledge of the treatment effect cutoff.

\vspace{5mm}
I proposed an estimator that:
    \begin{enumerate}
        \item Relaxes this assumption
        \item Allows estimation of the treatment effect curve instead of average effect
    \end{enumerate}

\vspace{5mm}
\begin{center}
\color{alice}{Thank you!}
\end{center}
    
\end{frame}



% ----------------------------------------------------------------------------
\begin{frame}[allowframebreaks]{References}
    \printbibliography
\end{frame}
% ------------------------------------------------------------------------------

\end{document}